\documentclass{article}
 
\usepackage{listings}
\usepackage{url}
 
\title{ENGS 116 Project Proposal}
\author{Gabe Stocco and Cory Cornelius}

\begin{document}
\maketitle
\begin{abstract}
In November of 2007, the National Institute for Standards and Technology (NIST) announced the request for submission of candidate algorithms for the Secure Hash Algorithm 3 (SHA-3).
Because of recent attacks against the current standards, NIST ``decided that it is prudent to develop a new hash algorithm to augment and revise FIPS 180–2'' \cite{Sha3Request}.

As the second round of candidates were recently chosen in September, we decided to choose few of these second-round candidates--CubeHash \cite{CubeHash}, Skein \cite{Skein}, and ECHO \cite{ECHO}--and see if it is feasible to implement them in a GPU computing architecture.
In particular, we chose use to implement them in Open Computing Language (OpenCL), which provides a framework for executing ``kernels'' on a variety of devices (e.g., CPUs, GPUs, etc.).

% FIXME: Add summary of findings.

\end{abstract}
 
\section*{Introduction}
Hash functions are a critical tool in computer security.
A hash function allows for fingerprinting and verification of a piece of data by running a deterministic algorithm over a set of data with the property that if any slight changes are made to the data a very different hash will result.
 
Recently, a number of popular and standard hash algorithms have been successfully attacked.
The widely used MD5 hashing algorithm has been demonstrated to be quite insecure \cite{Nostradamus}.
It has been shown to be cryptographically weak because it is not collision resistant and vulnerable to a kind of preimage attack--that is, it is very easy to create two arbitrary meaningful documents with the same hash.
In such a situation applications as varied as secure document downloading (using hashes as verification that you've received the right file) to online banking (hashes are used to verify security certificates for websites) can be compromised.
In addition, there have been recent serious attacks against one of the current standard hash algorithm, the Secure Hash Algorithm 1 (SHA-1) \cite{Sha1Collisions}.
Because SHA-2 is structurally similar to SHA-1 cryptographers believe SHA-2 will fall just as SHA-1 has.
 
As a result of these developments the National Institute of Standards and Technology (NIST) is currently running a competition to choose a new secure hash algorithm (SHA-3) \cite{Sha3Request}.
NIST is running this competition similarly to how they ran the competition for the new Advanced Encryption Standard (AES).
In particular, while a proposed algorithm might be stronger than every other proposed algorithm, the algorithm is useless if it takes a long time to execute.
Thus there is a requirement, as there similarly was in the AES competition, for the authors of the algorithm to provide documentation on the computational efficiency of the algorithm.
NIST will analyze ``each submission's optimized implementations on a variety of platforms as specified in Section 6.B, and for a variety of input message lengths.'' \cite{Sha3Request}.
Since computational efficiency will be an important part of the decision in choosing SHA-3 some authors of proposed algorithms have taken great care to design their algorithms in such a way to be exploit parallelism where ever possible.


\section*{Approach}
Because we are not skilled cryptographers and others are already analyzing these algorithms \cite{Sha3Zoo}, we will instead explore the runtime speed of a subset of the round 2 SHA-3 candidates.
We plan to evaluate the computational efficiency of the following SHA-3 candidates: Skein, CubeHash and Echo.
We choose Skein and CubeHash primarily based on the popularity of the authors in the cryptographic community, and we choose Echo because it is based upon AES.
While the each of these algorithms already have general purpose implementations we plan to implement them for execution on a GPU using the OpenCL and/or CUDA frameworks.
There is currently a wealth of information for FPGA and ASIC implementations of these algorithms, however we are unaware of OpenCL/CUDA implementations of any of the SHA-3 candidates \cite{Sha3Hardware}.
Our hypothesis is that some of these algorithms will benefit from parallel execution while others may not.
Thus, we will compare out runtime results on the GPU with runtime results on a CPU.
Our reference machine will be a Mac Pro an ATI Radeon HD 4870.
 
 
\section*{Algorithms}
We chose to implement SHA1 in OpenCL as a baseline to see how well the NIST competitors have been able to include parallelism in their designs.  Additionally, we implemented the CubeHash, ECHO, and Skein algorithms.

\subsection*{CubeHash}

CubeHash is perhaps the simplest SHA-3 candidate and, as such, as garnered significant attention.
CubeHash can be parameterized on the number of rounds $r$, the number of bytes per block $b$, and the number of output bits $h$.
For our tests, we chose $r=8$, $b=1$, and $h=512$ primarily because these were the recommended parameters specified by the author \cite{CubeHash-spec}.

\subsection*{ECHO}
ECHO is based largely on the AES block cipher chosen by NIST to replace the defunct Data Encryption Standard as an official federal government standard in the United States.
ECHO provides a number of advantages over other hash functions in the competition because is built upon the AES block cipher.
In particular, Intel chips starting with the current Nehalem core will have built in AES hardware support - allowing for three times faster operation than previous general purpose CPU chips.\cite{Westmere}
In addition to receiving performance enhancements from any AES optimizations built into hardware, testing ECHO on a GPU should also give some indication of how effective running AES as a GPGPU program would be.
Preliminary results from researchers have demonstrated up to a 2\cite{Harrison}-20x\cite{Manavski}  increase in encryption rate throughput by using the GPU as a encryption co-processor.
However, there is good reason to doubt the 20x figure as it appears that it was achieved through a naive implementation of AES using Electronic Code Book (ECB) mode.
Unlike more secure encryption modes, ECB mode encrypts data in finite sized chunks allowing for easy data parallelization.  
However, ECB mode is largely insecure for many use cases.\cite{CodeBook}
The only parameter to specify is the hashlength - we chose 512 bits for our tests.

\subsection*{Skein}



\section*{Implementation}

\subsection*{CubeHash}

\subsection*{Echo}

As part of the submissions materials to the NIST competition the ECHO team submitted a 32 bit optimized version, a 64-bit optimized version and a reference implementation.  
We chose to base our OpenCL implementation on the reference implementation, figuring that it would be the most likely to work with little modification.

\subsection*{Skein}


\section*{Evaluation}

\subsection*{CubeHash}

\subsection*{Echo}

\subsection*{Skein}

\section*{Conclusion}

\begin{thebibliography}{9}
  \bibitem{Sha3Request} ``Announcing Request for Candidate Algorithm Nominations for a New Cryptographic Hash Algorithm (SHAÐ3) Family.'' Federal Registrar, Vol. 72, No. 212 (November 2007), pp. 62212-62220. Available at \url{http://csrc.nist.gov/groups/ST/hash/documents/FR_Notice_Nov07.pdf}.
  \bibitem{Nostradamus} Marc Stevens, Arjen K. Lenstra, Benne de Weger. ``Predicting the winner of the 2008 US Presidential Elections using a Sony PlayStation 3.'' November 30, 2007. Available at \url{http://www.win.tue.nl/hashclash/Nostradamus/}
  \bibitem{Sha1Collisions} Xaoyun Wang, Yiqun Lisa Yin, Hongbo Yu. ``Finding Collisions in the Full SHA-1.'' Lecture Notes in Computer Science, Vol. 3621 (November 2005), pp. 17-36
  \bibitem{Sha3Zoo} ``The SHA-3 Zoo.'' Available at \url{http://ehash.iaik.tugraz.at/wiki/The_SHA-3_Zoo}
  \bibitem{Sha3Hardware} ``SHA-3 Hardware Implementations.'' Available at \url{http://ehash.iaik.tugraz.at/wiki/SHA-3_Hardware_Implementations}
  \bibitem{Westmere} ``Intel Nehalem (microarchitecture).'' Available at \url{http://en.wikipedia.org/wiki/Intel_Westmere}
  \bibitem{CubeHash} ``CubeHash: a simple hash function.'' Available at \url{http://cubehash.cr.yp.to/}
  \bibitem{CubeHash-spec} Bernstein, Daniel J. ``CubeHash specification (2.B.1).'' Available at \url{http://cubehash.cr.yp.to/submission/spec.pdf}.
  \bibitem{Skein} ``The Skein Hash Function Family.'' Available at \url{http://www.skein-hash.info/} and \url{http://www.schneier.com/skein.html}
  \bibitem{ECHO} ``ECHO Hash Function.'' Available at \url{http://crypto.rd.francetelecom.com/echo/}.
  \bibitem{Harrison} Owen Harrison and John Waldron. ``AES Encryption Implementation and Analysis on Commodity Graphics Processing Units.'' Available at \url{https://www.cs.tcd.ie/~harrisoo/publications/AES_On_GPU.pdf}.
  \bibitem{CodeBook} ``Block cipher modes of operation'' Available at \url{http://en.wikipedia.org/wiki/Block_cipher_modes_of_operation#Electronic_codebook_.28ECB.29}.
  \bibitem{Manavski}
\end{thebibliography}

\end{document}