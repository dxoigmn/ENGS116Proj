\documentclass{article}

\usepackage{listings}
\usepackage{url}
\usepackage[nohead]{geometry} 

\title{ENGS 116 Project Proposal}
\author{Gabriel Stocco and Cory Cornelius}
\date{14 October 2009}
\begin{document}

\maketitle

\section*{Introduction}
Recently, a number of widely used and popular cryptographic hash functions have been successfully attacked.  In particular, the popular MD5 hash function has been demonstrated to be completely broken, allowing for crafted collisions to generate forged documents which then appear genuine.\cite{Nostradamus}

\begin{quotation}A cryptographic hash function is a deterministic procedure that takes an arbitrary block of data and returns a fixed-size bit string, the (cryptographic) hash value, such that an accidental or intentional change to the data will change the hash value.\cite{HashDef}
\end{quotation}

The MD5 hashing algorithm has been found to be vulnerable to a kind of pre-image attack -- that is, it is possible to create custom documents which produce a known hash, greatly reducing the complexity of creating forged documents to easily realizable levels.
In such a situation, applications as varied as secure document downloading (using hashes as verification that you've received the right file) and online banking (hashes are used to verify security certificates to for secure communication with websites) can be compromised.
In addition, there have been recent serious attacks against one of the current national standard hash algorithms, the Secure Hash Algorithm 1 (SHA-1).\cite{Sha1Collisions}

As a result of these developments the National Institute of Standards and Technology (NIST) is currently running a competition to choose a new secure hash algorithm called SHA-3.\cite{Sha3Request}
NIST earlier ran a competition to choose the Advanced Encryption Standard (AES) to replace the widely used Data Encryption Standard.
Among the requirements for the new SHA-3 standard is computational efficiency - NIST will analyze ``each submission’s optimized implementations on a variety of platforms as specified in Section 6.B, and for a variety of input message lengths.''\cite{Sha3Request}
Since computational efficiency will be an important part of the decision in choosing SHA-3, some authors of proposed algorithms have taken great care to design their algorithms in such a way to be exploit parallelism where ever possible.
We see the greatest amount of potential for parallelism in Modern GPUs which can have hundreds of parallel processing elements.

\section*{Approach}
Because we are not skilled cryptographers and others are already analyzing these algorithms \cite{Sha3Zoo}, we will instead explore the runtime speed of a subset of the round 2 SHA-3 candidates.
We plan to evaluate the computational efficiency of the following SHA-3 candidates: Skein, CubeHash and Echo.
Skein and CubeHash were chosen because of the popularity of the authors in the cryptographic community, and Echo was chosen because it is based on AES.
While the each of these algorithms already have general purpose implementations, we will implement them for execution on a GPU using the OpenCL and/or CUDA frameworks.
There is currently a wealth of information for FPGA and ASIC implementations of these algorithms, but we are unaware of OpenCL/CUDA implementations of any of the SHA-3 candidates \cite{Sha3Hardware}.
Our hypothesis is that some of these algorithms will benefit from parallel execution while others may not.  
Thus, we will compare our runtime results on the GPU with runtime results on a CPU.
Our reference machine will be a Mac Pro with an ATI Radeon HD 4870 and 8-cpu cores.

\section*{Plan}
\begin{description}
  \item[Week 1] Learn and experiment with OpenCL/CUDA APIs.
  \item[Week 2-3] Implement algorithms such that they comply with the test vectors as specified by the authors.
  \item[Week 4] Evaluate the runtime of our implementations as compared to the reference implementations on a CPU. Prepare project presentation and report.
\end{description}

\begin{thebibliography}{9}
	\bibitem{Nostradamus} Marc Stevens, Arjen K. Lenstra, Benne de Weger. ``Predicting the winner of the 2008 US Presidential Elections using a Sony PlayStation 3.'' November 30, 2007. Available at \url{http://www.win.tue.nl/hashclash/Nostradamus/}
	\bibitem{HashDef} ``Cryptographic Hash Function.'' Available at \url{http://en.wikipedia.org/wiki/Cryptographic_hash_function}
  \bibitem{Sha1Collisions} Xaoyun Wang, Yiqun Lisa Yin, Hongbo Yu. ``Finding Collisions in the Full SHA-1.'' Lecture Notes in Computer Science, Vol. 3621 (November 2005), pp. 17-36
\bibitem{Sha3Request} ``Announcing Request for Candidate Algorithm Nominations for a New Cryptographic Hash Algorithm (SHA–3) Family.'' Federal Registrar, Vol. 72, No. 212 (November 2007), pp. 62212-62220. Available at \url{http://csrc.nist.gov/groups/ST/hash/documents/FR_Notice_Nov07.pdf}.
  \bibitem{Sha3Zoo} ``The SHA-3 Zoo.'' Available at \url{http://ehash.iaik.tugraz.at/wiki/The_SHA-3_Zoo}
  \bibitem{Sha3Hardware} ``SHA-3 Hardware Implementations.'' Available at \url{http://ehash.iaik.tugraz.at/wiki/SHA-3_Hardware_Implementations}
\end{thebibliography}

\end{document}
